\section{Выполнение задания}

\subsection*{Общее описание}

\textbf{Фильм Хэнкок, часть 1.} 

Главный герой фильма --- Джон Хэнкок, 
супергерой, а точнее, он обладает супер способностями 
и благодаря им по настроению спасает друзей. 

Супергероем города или страны его не назовешь, 
так как народ не любит Хэнкока, а точнее, 
не любит методы его дерзких "спасений", 
когда он крушит все вокруг. 

Хэнкок может спасти человека, но при этом повредить асфальт, 
машины или дома, потому что он неаккуратен, 
да и, в целом, опрятность совершенных им действий 
не сильно то его и заботит. 

В итоге супергерой приносит миллионный ущерб городу. 
Хэнкок не может вспомнить своего прошлого, с кем он был, 
что делал, так как много десятков лет назад получил ранение в голову. 
К тому же, а возможно, и от ранения, 
Хэнкок изрядно выпивает. 
Бутылка виски или коньяка у него всегда под рукой и от этого, 
конечно, народ еще больше его не любит.

В один из дней, Хэнкок снова спасает человека, 
все также разрушив часть инфраструктуры района. 
Вокруг него собирается толпа людей, которая кричит, 
как его ненавидит и, наконец, один человек 
(которого Хэнкок только что спас) говорит о том, 
что он наоборот благодарен супергерою. 
После, спасённый Рэй приглашает Хэнкока к себе в гости на ужин
и за обедом предлагает свою помощь Хэнкоку.
А именно --- сделать из него настоящего супергероя, 
которого любит народ, уважает полиция, который полезен миру.

После раздумий Хэнкок соглашается, и они с Рэем начинают работу. 
Одна из задач Хэнкока — перестать бесконично пить. 
Рэй мотивирует Хэнкока отказаться от такого огромного количества
употребляемого алкоголя в день, 
давая этому разумные объяснения, 
а жена Рэя часто обвиняет Хэнкока, 
упрекая его в употреблении алкоголя перед ее ребенком 
и в появлении героя дома у семьи Рэя и Мэри в нетрезвом виде. 
Упрекает в пьянстве Хэнкока и народ города.

\subsection*{Цели и мотивы участников}

\textbf{Игра}. Алкоголик.

\textbf{Тезис}. Посмотрим, сможешь ли ты остановить мое поведение, которое никому не нравится

\textbf{Цель}. Самобичевание. Провокация.

Роли.
\begin{itemize}
    \item Алкоголик — Хэнкок
    \item Преследователь — жена Рэя, Мэри
    \item Спасатель — Рэй
\end{itemize}

\textbf{Динамика}. Оральная депривация

\subsection*{Социальный уровень и «перевод» на психологический; вознаграждения}

\begin{enumerate}
    \item Социальная парадигма.
    
    Взрослый — Взрослый.
    
    Взрослый: “Скажи, что ты на самом деле обо мне думаешь, чтобы я сам мог на этой основе бросить пить” или “Помоги мне прекратить пить”
    Взрослый: “Я буду с тобой честен” или “Я помогу тебе бросить пить”
    
    \item Психологическая парадигма.
    
    Родитель — Ребенок.
    
    Ребенок: “Попробуй поймать меня” или “Попробуй убедить меня”
    Родитель: “Ты должен прекратить пить, потому что…”
    
    \item Ходы.
     
    Провокация — обвинение
    
    Желание помочь — отрицание, отсутствие интереса слушать 
    
    Снисходительность — гнев
    
    \item Вознаграждения.
    
    \textit{Внутреннее психологическое} — употребление алкоголя как протест, как утешение или, как банальное удовлетворение желание, или самобичевание
    
    \textit{Внешнее психологическое} — возможность избежать длительного контакта с людьми, так как у них идет отвращение от общения с таким человеком 
    
    \textit{Внутреннее социальное} — внутренняя провокация, которая звучит, как “Посмотрим, сможешь ли ты меня остановить”
    
    \textit{Экзистенциальное} — “Никому нет до меня дела”
\end{enumerate}

\subsection*{Антитезис}

Занять позицию Взрослого (со стороны народа-преследователя, Мэри-преследователя и Рэя-спасателя) и отказаться участвовать в игре в любой роли или оставить только Рэя-спасителя. 

\subsection*{Итог игры}

В итоге, Хэнкок не отказался от выпивки совсем, однако перестал пить при детях вблизи, то есть игра, в целом, продолжается. Как минимум потому, что роли спасителя, преследователя и самого алкоголика остались. 
